\documentclass[11pt]{article}
\usepackage[margin=1in]{geometry}
\usepackage[T1]{fontenc}
\usepackage{lmodern}

\title{Technical Scaffolding for Interdisciplinary Collaboration in Serious Games}
\author{Ali Rezaei}
\date{February 4, 2026}

\begin{document}
\maketitle

\section{Project role and context}
This case study covers my technical consulting work on ``Mudbricks and Pixels,''
an ongoing interdisciplinary project (September 2025 to present) that simulates
Adobe and mudbrick construction processes. The project aims to democratize
architectural knowledge by building a VR and web-based program that teaches
fundamental construction skills to women in different parts of the world. The
core team included two undergraduate design students, a supervising professor,
and me as the lead technical consultant.

My main responsibility was to create a development environment that let
non-technical stakeholders participate directly in the Unity workflow. Rather
than only implementing requests, I acted as a technical translator and built
infrastructure that let designers bring assets, from custom models to 3D scans
of real mudbrick sites---into a physics-driven environment without requiring
them to be fluent in C\#.

\section{Collaboration challenges: the vocabulary of interaction}
Early collaboration was slowed by a vocabulary mismatch between engineering
specs and design goals. Designers focused on narrative and tactile realism:
how a mudbrick should feel heavy, or how a surface should respond to tools.
Unity's physics engine, by contrast, relies on mass, drag, and friction.

To the designers, the physics system felt chaotic rather than creative, and
initial attempts to communicate with raw documentation created a barrier to
entry. To support the instructional goals of the project, we had to demystify
``physics simulation'' and reframe it as a set of predictable, adjustable
creative controls.

\section{Methods: documentation as infrastructure}
I addressed the gap using pedagogical documentation and analogical scaffolding.
We built internal guides that explained abstract behavior with metaphors the
design students already understood.

\subsection{Metaphorical translation}
In the internal handbook, \textit{Working with Physics in Unity}, I avoided
vector math and leaned on physical analogies. For example, ``drag'' became an
``air brake'' for objects rather than an air-resistance coefficient. A
``kinematic'' rigidbody was described as a ``ghost driver'' that stays on rails
and can push other objects but is not driven by forces. Colliders were framed
as ``protective bubbles'' that define physical shape.

\subsection{Inspector-first workflow}
I promoted a workflow where the Unity Inspector was the primary tool for
designers. The documentation emphasized that physics can be ``set and forget'':
gravity and collisions are handled automatically once values are configured.
This let designers use Physics Materials to tune ``bounciness'' and ``friction''
to distinguish surfaces (for example, ``ice'' versus ``sandpaper'') without
writing code.

\subsection{Scaffolded interaction models}
We also distinguished between ``teleporting'' objects (direct transform edits)
and ``pushing'' them with forces. The distinction mattered most for VR, where
natural motion is key to immersion.

\section{Results}
Design students were able to populate scenes with interactive assets and
configure Rigidbodies and Colliders to match the physical properties of mud and
stone. Communication improved as the team adopted the shared metaphorical
language. Conversations shifted from vague complaints about ``glitchy'' motion
to precise requests about collision layers and trigger zones.

\section{Lessons learned and recommendations}
Based on the ``Mudbricks and Pixels'' collaboration, these practices are
recommended for engineers working with creative teams in digital heritage
contexts:

\begin{itemize}
  \item Establish a ``pidgin'' language early by mapping technical terms to
    domain-specific metaphors (for example, triggers as sensor regions or
    tripwires).
  \item Treat documentation as a product. Internal guides should be delivered
    alongside code and use a friendly, second-person tone to reduce anxiety.
  \item Scaffold the ``happy path'' with tools and presets so designers can
    focus on creative variables. Explain common misconceptions, such as mass
    affecting collisions but not fall speed.
  \item Prioritize one-click feedback. Designers need immediate visual
    confirmation, so encourage frequent use of Play Mode.
  \item Layer complexity over time. Start with basic collisions (solid objects)
    before introducing triggers (sensors) or collision layers.
  \item Contextualize constraints. Frame settings like Fixed Timestep or Solver
    Iterations as trade-offs between smoothness and accuracy, not math problems.
\end{itemize}

\section{Limitations}
This reflection is based on qualitative observation, and the project is still
ongoing. We did not run formal pre- and post-tests on technical literacy. The
collaboration also depended on the specific constraints of ``Mudbricks and
Pixels,'' which prioritized physics-based spatial interaction over complex
algorithmic logic. Projects that require heavy custom scripting may encounter
different hurdles when bridging the engineer-designer divide.

\end{document}
